\documentclass{article}
\usepackage[top=1in, bottom=1in, left=1in, right=1in]{geometry}

\usepackage{fullpage}
\usepackage{url}

\author{
Luke Anderson, Jon Gjengset, Jeevana Inala, and Andrew Wang \\
\texttt{\{lukea,jfrg,jinala,wangaj\}@mit.edu}
}
\title{Concolic Execution for Django Applications}
%\subtitle{6.858 Project Report}
\date{}
\begin{document}

\maketitle
\section{Motivation}
Concolic execution systems allow developers to verify that invariants in their
applications are not violated no matter what input is given by a user. This was
demonstrated in Lab 3, where the Z3 solver was used to find inputs that would
trigger inconsistencies in zoobar balances. Unfortunately, the lab's framework
is written specifically for Zoobar, and would therefore not work for other
applications without substantial modifications.  Invariant checking is useful
to a wide range of applications, and thus we decided to make our 6.858 final
project building a generic concolic execution interface for any Django-based
web application.

\section{Summary of results}

Our first step towards running concolic invariant checking on Django
applications was to build an application we could test. For this, we decided to
port the original lab 3 Zoobar application. By preserving as much of the
original application structure and code as, while still staying withing the
confines of idiomatic Django application construction, this enabled us to test
our concolic framework against the same set of invariants used by the Lab 3
version. Specifically, we demonstrate the same functionality as the original
application, and show that  the same bugs are found in our invariant checker.
The code is available at \url{https://github.com/jonhoo/django-zoobar}.

The next step was to construct a version of the Lab 3 concolic execution
framework that cold support Django applications. The biggest change
we made was undoubtedly to make concolic inputs be preserved throughout the
complex internals of Django, so that they reach the application logic
intact. This is implemented in \texttt{symex/symdjango.py}, which replaces
\texttt{symex/symflask.py} from Lab 3. In particular, we found that our
concolic inputs were being lost in the parsing of URLs and POST form data,
which is deeply integrated into Django's codebase. Replacing those unsupported
operations allowed the concolic values to pass through to Zoobar without losing
their symbolic half.

Since real web apps may have large databases, we also considered various
options to exercise good coverage while avoiding an exponential increase in the
number of paths to check. We ended up implementing three different approaches
in \texttt{symex/symqueryset.py}: In addition to searching all database items as
in Lab 3, we also tried mutating the WHERE clause of SQL queries to
capture subtle edge cases, and dynamically inserting concolic values
in the database itself. These approaches all have their pros and cons,
and in the end, we allow the developer to choose which one to use.

With an eye towards large applications, and with the complexity of Django in mind,
we wanted to optimize the constraint-checking framework as much
as possible. Several optimizations were added to \texttt{symex/fuzzer.py}. Some
of these were drawn from \textit{KLEE}~(Cadar, Dunbar, and Engler 2008), such
as removing implied constraints, and keeping a counter-example cache.
The most impactful optimization, though, was to avoid duplicate inputs
arising from different path conditions all yielding the same constraints.
The concolic checker is available at \url{https://github.com/jonhoo/django-coex}.

An example of how to initialize the concolic framework, as well as how to issue
concolic requests can be seen in \texttt{check-symex-zoobar.py}, which was
adapted from Lab 3. For reasons explained in the Details section below, the
first step is to make explicit the URLs to test, and the database query method
to use. The test function and invariants are then defined similarly to in Lab
3, and the concolic tester is invoked at the bottom of the file. Using the
default settings of checking all database objects and turning on all
optimizations, the framework runs 48 iterations in approximately 30 seconds.

We also ran our concolic execution system on gradapply; the MIT graduate
applications website, which is a real application, and substantially more
complex than zoobar. Since gradapply is
written for an older version of Django (1.6), we first extended our
system to support version 1.6. Then it was straightforward to adapt
the checking script. The system tests six different URL paths, and
completes in 77 iterations. Given the project time constraints, we did not
write invariants and find bugs in gradapply, though we believe this should be
straightforward now that the checker runs correctly.

Contrary to our initial thoughts, we did not have to modify the
Django codebase. This was made possible by the dynamic nature of
Python, which allowed us to overwrite or bypass methods that were
incompatible with concolic execution during runtime.

\section{Implementation details}

For each component of the project, we now go into greater detail explaining our
design decisions and the challenges we faced.

\subsection{Preserving concolic values}

\begin{enumerate}
\item URL parsing in Django is based on nested regular expressions
  that are used both to determine which view should be invoked and to do
  reverse URL lookups.
  Unfortunately, regular expressions are not supported by Z3, so we
  needed another way to make the choice of view and the input
  parameters concolic. The simplest solution we arrived at was to
  let the developer specify a mapping of views to URLs, allowing them to
  separately pass in
  named parameters. This was favored over calling all the views in an
  application directly, which is more automated, but would also
  bypass all the Django middleware modules that many applications
  rely on.

\item Web applications receive input from a number of sources: in-URL
  parameters, GET/POST parameters, file uploads, cookies, etc., and
  there is no simple way to automatically determine which of these an
  application cares about. For the scope of the project, we took the
  approach of requiring the developer to explicitly construct the
  datasets with concolic values and pass them in whenever a URL points
  to a relevant view. Future work could reduce the developer's burden
  by letting the framework dynamically discover the format of these inputs.

\item So far, the concolic execution framework supports only a
  fraction of the vast set of Python operations available to Django
  and web apps. For Zoobar, we spent considerable time locating
  unsupported operations in the Django codebase, such as POST form
  data serialization, and URL resolution, and finding ways to circumvent those
  operations. It is likely that there are still several corner cases
  in the Django codebase where concolic values are lost, but the best
  way to identify them would be additional testing on a diverse set of
  applications.
\end{enumerate}

\subsection{Handling database queries}

We aim to achieve good coverage
by exposing the internal database query decisions to the concolic checker, while retaining
efficiency on large datasets. This goal is difficult to achieve with a single
algorithm, as different datasets may or may not expose different branches in
applications. The ideal solution is likely a hybrid approach that dynamically adapts to
the situation at hand, but for now, we discuss the strengths and weaknesses of
the individual database enumeration approaches we developed.

\begin{enumerate}
\item \textit{All}:
  This strategy mirrors Lab 3: We iterate over all rows and compare each to the lookup
  key.
  We extend this 
  to support lookups with multiple keys, and lookups across
  related tables.
  This approach is relatively simple, and generally provides good
  coverage. However, it potentially re-tests the same branches,
  which is wasteful, particularly for large datasets. Further, there are situations (described
  below in \textit{SQL}) where branches will not be covered.

\item \textit{Mutation}:
  This strategiry is based on \textit{ConSMutate}~(Sarkar,
  et.\ al. 2012): when a query is made, we extract the WHERE clause,
  then mutate it by changing the relational operator, e.g.\, 
  $zoobars > 10$ is one possible mutation of $zoobars \geq 10$.
  If running the mutated query yields a new result, then we
  add the symmetric difference of the two conditions as a new path
  constraint. The hope is that the new constraint will lead to a
  concrete value that explores an untouched branch of the code. This
  approach has lower coverage but can potentially catch common
  programmer errors.

\item \textit{SQL}:
  This strategy attempts to maximize the branch coverage in applications. To
  exemplify, cosnider the two obstacles that may prevent us from exploring both
  branches of the IF statement below:
  1) \textit{users.person.zoobars} is not
  concolic; 2) there may be no user with $>$ 5 zoobars.
  \begin{verbatim}user = User.objects.get(username='bob')
if user.person.zoobars > 5: # do something
else: # do something else\end{verbatim}
  Our solution is to map each database object's properties to new concolic
  values, e.g., creating a new concolic int for
  \textit{user.person.zoobars} when \textit{user} is assigned. We also
  automatically create new database entries when Z3 generates
  concrete values that are absent, e.g., before we try an input where
  \textit{user.person.zoobars} is 7, we ensure that there is a
  database entry with that value. Compared to \textit{All}, we
  achieve slightly better coverage on zoobar in even fewer iterations.
  The downside is that changing the database during a test makes it harder to
  write invariants.
\end{enumerate}

\subsection{Concolic optimizations}

Concolic execution invariant checking time is dominated by running the Z3
solver. Runtime can thus be significantly improved by reducing either the
overall number of iterations, or by reducing the time it takes to solve each Z3
instance.

\begin{enumerate}
\item To reduce the number of iterations, we keep a record of previous
  inputs to the application, in addition to previous paths, as in Lab 3. Sometimes,
  different path conditions yield the same solution, in which case
  the web app would process an identical request. Avoiding this
  situation reduced zoobar's iterations from 89 to 81. (These numbers
  do not reflect the database querying strategies being employed
  to make the concolic framework more efficient.) We suspect the impact of this
  approach will be more noticeable for larger, and more complex applications.

\item To reduce the complexity of the constraint programs sent to Z3,
  we detect when some subet of the constraints are implied by others. This is done each
  time we extend a path condition by a new constraint. Note that
  determining implication itself is a constraint program, so to avoid
  running Z3 recursively on these, we determined a set of syntactic rules, which
  capture most implications in Zoobar that Z3 would have found. This
  further reduced the number of iterations to 75.

\item Sometimes a constraint program may not need to be solved if a
  solution has already been recorded in a counter-example cache. The
  cache does not eliminate any iterations, but the amortized runtime
  benefit of finding a solution in the cache may outweigh the cost of
  having to run the solver anyway. On Zoobar, the counter-example
  cache was successfully employed 28 times, reducing runtime noticeably simply
  by reducing the number of costly Z3 solver calls.
\end{enumerate}

\end{document}
