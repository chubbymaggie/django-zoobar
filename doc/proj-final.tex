\documentclass{scrartcl}

\usepackage{fullpage}
\usepackage{url}

\author{Luke Anderson, Jon Gjengset, Jeevana Inala, and Andrew Wang \\
	\texttt{\{lukea,jfrg,jinala,wangaj\}@mit.edu}
}
\title{6.858 Project Write-up}
\subtitle{Concolic Execution for Django Applications}
\date{\today}

\begin{document}
\maketitle

\section{Project background}
Concolic execution systems give developers the power to verify crucial security
invariants in their applications. The third lab exercise in 6.858 uses the z3
solver to implement concolic execution for the Zoobar python application.
Unfortunately, the framework used for the lab is written specifically for
Zoobar, and is unlikely to be directly compatible with other applications. We
therefore set out to extend the lab 3 concolic execution implementation to one
that operates on web applications built on top of the popular Django framework.
This required finding ways of preserving concolic execution information during
the execution of internal framework code in Django, as well as developing
necessary optimizations to deal with the additional complexity provided by the
many features Django has to offer.

\section{Challenges}
Over the course of developing this framework, we came across a number of
challenges; both anticipated and unexpected. We list here the more important
ones, and how we decided to tackle them:
\begin{enumerate}
	\item URL parsing in Django is based on nested regular expressions that
		are used both to determine which application method (``view''
		in Django terminology) should be invoked, and to do reverse URL
		lookups. These patterns can get very complex, and often involve
		dynamic parameters that may be passed as input to the
		application. Unfortunately, regular expressions are not
		supported by z3 (and support for them would probably prove
		difficule to add), so we needed to come up with a different
		scheme for making the choice of view, as well as parameters
		passed to the view, concolic.

		There are many ways we could have chosen to approach this
		problem. One that is fairly straightforward is to dynamically
		resolve the list of all views in an application, and call them
		directly, ignoring URLs altogether.  The downside of this
		approach is that this would bypass all of Django's middleware
		modules (handlers that the request is passed through before it
		is passed to the appropriate view), which many applications
		rely heavily on for features such as session management and
		CSRF protection.

		Instead, we chose to have the developer explicitly specify
		which URLs map to which views, along with a Python dictionary
		of named parameters. This is not very elegant, and requires
		that the developers enumerates all views they want to exercise,
		but it is significantly easier and less error-prone to
		implement than any kind of dynamic discovery scheme.
	\item Finding a good way of exposing decisions made inside the database
		to the concolic execution framework was difficult. While the
		approach taken in Lab 3 where we enumerate all entries in a
		table and force a branch on every row from the table does
		ensure that all application branches are tested, it is also
		prohibitively expensive for applications that may have large
		datasets already present in their databases. Furthermore, it is
		wasteful, as most rows are likely to not yield different
		application behaviour, and thus unnecessarily increase the
		search space. Furthermore, all database operations would need
		to be reimplemented for this technique to correctly preserve
		concolic values for all possible Database lookups.

		Instead, we opted for using a faster, but also weaker approach
		by default, with the option to fall back to the exhaustive Lab
		3 technique. Whenever a query to the database is constructed
		where one of the conditions uses a concolic value, we construct
		an artificial branch in which the database condition is
		negated. This will force the concolic execution framework to
		try at least two different results for that query, potentially
		finding more application branches. Obviously, this approach
		will potentially miss branches, particularly if only a few rows
		in a database have a set of values that would reach some
		branch, but it is also significantly faster for applications
		with large datasets.

		We have some ideas about how this search might be improved by
		specifically looking for other rows that do not match the given
		condition, and where other database values are different, but
		did not have time to implement these by the project deadline.
		Instead, we leave these improvements as future work. In the
		meantime, we suggest developers test their application with a
		test database with exhaustive search, as well as with a copy of
		the full database using the reduced search.
	
	\item Django, and applications written atop it, use a significant
		subset of the operations and types provided by the Python
		language. The concolic execution framework only supports a
		subset of these (regular expressions and floating point numbers
		are noteworthy examples), and thus the concolic nature of
		values can easily get lost for today's complex web
		applications. We spent considerable time determining exactly
		which part of the Django codebase was using unsupported
		operations (e.g. URL resolution), and coming up with ways of
		circumventing those operations. Many of these were deeply
		integrated into the Django codebase, such as the parsing of
		POST form data and URLs. There are likely still parts of the
		Django codebase where concolic values would be lost, and
		testing the framework on a more diverse set of applications
		would be necessary to expose all of the corner-cases.
	
	\item Web applications receive input from a number of sources; in-URL
		parameters, GET parameters, POST form parameters, file uploads,
		environment variables, and cookies just to name a few.
		Furthermore, there is no straightforward way of determining
		which of these a given application view uses, an in particular,
		which values it cares about. We have taken the approach of
		requiring the developer to explicitly construct the datasets
		(with concolic values where appropriate) to give to
		applications, and to pass these datasets in whenever a URL
		points to a relevant view. While this is inconvenient (the
		developer has to know and enumerate what data can be passed to
		every view of interest), it was the only feasible way we could
		imagine doing it in the available timeframe. Future work should
		include finding better ways of determining view inputs, or at
		the very least, simplifying the developer's job of telling the
		framework about the format of these inputs.
	
	\item The framework from Lab 3 only tests invariants across single
		requests. Realistically however, a sequence of requests may be
		required for a particular bug to be exposed. We thus had to add
		support for running multiple requests one after another, while
		considering the entire sequence of requests a single execution.
		This is primarily hard because of the exponential growth of the
		search space, as the number of branches for each request are
		multiplied together. We chose to add this feature to the
		framework for the sake of completeness, but have not had time
		to test or optimize this feature for any of our applications.
\end{enumerate}

\section{Results}
Over the course of the project, we have produced two codebases of interest.
First, we have ported the Stanford Zoobar application to Django. The code for
this can be found at \url{https://github.com/jonhoo/django-zoobar}. We have
intentionally attempted to preserve the application structure and code from the
original codebase as far as possible, both to make comparisons with the
original easier, as well as to ensure that the same set of bugs can be found in
both.

Second, we have modified and expanded the Lab 3 concolic execution framework to
support Django applications, as well added a testing script for the ported
version of the Zoobar application. The code for this can be found at
\url{https://github.com/jonhoo/django-coex}. Changes of particular interest are
the optimizations added to fuzzer.py, the introduction of symex/symdjango.py
which does all the song and dance necessary to get concolic values into Django
applications, and symex/symqueryset.py which implementes the database query
modifications needed for concolic execution as discussed above. It is also
worth noting that the changes to check-symex-zoobar.py are relatively
straightforward. The passing of request data has been moved out of symflask
(where it did not belong as it was application-dependent), and the URLs to test
has been made explicit, but all the invariant tests are pretty much the same,
as are the ways of invoking the checker.

Contrary to what we initially thought, we did not have to modify Django as a
part of this project. This was made possibly by the dynamic nature of Python,
which allowed us to patch running Django code, thus overwriting or bypassing
methods that were incompatible with concolic execution (see description of some
of these in the previous section).

\subsection{Concolic optimizations}

Because our goal is to deploy concolic execution on large, real-world
Django web applications, we would like the constraint-checking framework
to be as efficient as possible. To that end, we implemented the
following optimizations to the Lab 3 framework in fuzzy.py and measured
their performance.

\begin{itemize}
\item When a request has been processed through the web app and a path
condition is obtained, one identifies partial paths that branch off the
current path and puts them on the queue. In Lab 3, we were branching
both ``left'' and ``right'' on each element in the path. However, since
one of these is going to be a partial path of the current one, it adds
no new information. Not adding that path on the queue reduces the number
of iterations by half. In Zoobar, we went from 181 to 89 iterations.

\item We noticed that the concolic engine often ran inputs it had
already tried, even though we already maintain a cache of path
conditions previously solved. It turns out different path conditions
may yield the same solution, in which case they should be considered
equivalent. The solver computation cannot be avoided, but one can
eliminate processing an identical request in the web app. This further
reduced Zoobar's iterations from 89 to 81.

\item The KLEE paper mentions removing redundant constraints from a
constraint program to reduce the complexity for the solver. If a new
constraint implies an existing one, or vice versa, then the constraint
that is implied can be eliminated.
Determining implication in the general case is expensive, requiring a
solver. To avoid the cost of running Z3, we checked pairs of constraints
against several patterns of string constraints where one is known to
imply the other. We were able to detect all implications in Zoobar that
Z3 would have, and further reduced the number of iterations to 75.

\item A novel feature of KLEE is its use of the counter-example cache
that records previous solutions or no-goods, to be used as a heuristic
on future iterations. The cache does not eliminate any iterations, but
the runtime benefit of finding a solution in the cache may outweigh the
cost of looking up the cache and having to run the solver anyway.
The cache is implemented as a UBtree by [Hoffmann and Koehler],
which allows efficient subset and superset queries. On Zoobar, the
counter-example cache was successfully employed 28 times.
\end{itemize}

\end{document}
