\documentclass{scrartcl}

\usepackage{fullpage}
\usepackage{url}

\author{
Luke Anderson, Jon Gjengset, Jeevana Inala, and Andrew Wang \\
\texttt{\{lukea,jfrg,jinala,wangaj\}@mit.edu}
}
\title{6.858 Project Report}
\subtitle{Concolic Execution for Django Applications}
\date{\today}

\begin{document}
\maketitle


\section{Motivation}

Concolic execution systems allow developers to verify security
invariants in their applications. This was demonstrated in Lab 3,
where the Z3 solver was used to find inputs that would trigger
inconsistencies in zoobar balances. Unfortunately, the lab's
framework is written specifically for the zoobar web application, and
would be difficult to adapt for testing any other application. We
therefore set out to provide an interface to concolic execution that
works for any web app built on top of the popular Django framework.


\section{Summary of results}

Our work is contained within two code repositories. These are
publicly available at:
\begin{itemize}
\item \url{https://github.com/jonhoo/django-zoobar}
\item \url{https://github.com/jonhoo/django-coex}
\end{itemize}

The first repository is a port of the zoobar application to Django.
This enables us to test our concolic framework with the proven Lab 3
version. Specifically, we demonstrate the same functionality and that
the same bugs are found in our version. To that end, we have attempted
to preserve the original application structure and code as much as
possible.

The second repository contains the modifications to the Lab 3 concolic
execution framework that allow us to support Django applications. Most
importantly, concolic inputs have to be preserved throughout the
complex internals of Django, so that they reach the application logic
intact. This is achieved in symex/symdjango.py, which replaces
symex/symflask.py from Lab 3. We found that our concolic inputs were
being lost in the parsing of URLs and POST form data, which is deeply
integrated into Django's codebase. Replacing those unsupported
operations allowed the concolic values to pass through to zoobar.

Since real web apps may have large databases, we considered various
options to exercise good coverage without exploding the number of
paths. We ended up implementing three different approaches in
symex/symqueryset.py: In addition to searching all database items as
in Lab 3, we also tried mutating the WHERE clause of SQL queries to
capture subtle edge cases, and dynamically inserting concolic values
in the database itself. These approaches all have their pros and cons,
and in the end, we allow the developer to choose which one to use.

With an eye towards large applications and the complexity of Django,
we also wanted to optimize the constraint-checking framework as much
as possible. Several optimizations were added to symex/fuzzer.py. Some
of these were drawn from the KLEE paper (Cadar, Dunbar, and Engler
2008), such as removing implied constraints, and keeping a
counter-example cache. The most impactful optimization, though, was
to avoid duplicate inputs arising from different path conditions.

Use of our concolic framework by the developer is exemplified in
check-symex-zoobar.py, which was adapted from Lab 3. For reasons
explained in the Details section, the first step is to make explicit
the URLs to test and the database query method. Then, the test
function is defined similarly, using the same invariants as in Lab 3,
and the concolic tester is invoked on it. Using the default settings
of checking all database objects and turning on all optimizations, the
framework runs 48 iterations in approximately 30 seconds.

Finally, we also ran our concolic execution system on gradapply, which
is a real and a much complex website than zoobar. Since gradapply is
written in an older version of Django (1.6), we first extended our
system to support version 1.6. Then it was straightforward to adapt
the checking script. The system tests six different URL paths, and
completes in 77 iterations. In the time we had, we descoped writing
invariants and finding bugs in gradapply.

Contrary to what we initially thought, we did not have to modify the
Django codebase. This was made possible by the dynamic nature of
Python, which allowed us to overwrite or bypass methods that were
incompatible with concolic execution.


\section{Implementation details}

Discuss the challenges and the design choices made.

\subsection{Preserving concolic values}

\begin{enumerate}
\item URL parsing in Django is based on nested regular expressions that
  are used both to determine which application method (``view''
  in Django terminology) should be invoked, and to do reverse URL
  lookups. These patterns can get very complex, and often involve
  dynamic parameters that may be passed as input to the
  application. Unfortunately, regular expressions are not
  supported by Z3 (and support for them would probably prove
  difficult to add), so we needed to come up with a different
  scheme for making the choice of view, as well as parameters
  passed to the view, concolic.

  There are many ways we could have chosen to approach this
  problem. One that is fairly straightforward is to dynamically
  resolve the list of all views in an application, and call them
  directly, ignoring URLs altogether.  The downside of this
  approach is that this would bypass all of Django's middleware
  modules (handlers that the request is passed through before it
  is passed to the appropriate view), which many applications
  rely heavily on for features such as session management and
  CSRF protection.

  Instead, we chose to have the developer explicitly specify
  which URLs map to which views, along with a Python dictionary
  of named parameters. This is not very elegant, and requires
  that the developers enumerates all views they want to exercise,
  but it is significantly easier and less error-prone to
  implement than any kind of dynamic discovery scheme.

\item Finding a good way of exposing to the concolic execution framework
  decisions made inside the database proved challenging. We implemented 3
  approaches, described below, each with tradeoffs. In general, it seems
  that higher code coverage typically comes at the cost of efficiency, and an
  ideal solution may be a hybrid approach that can detect situations in which
  one algorithm is more effective than the others and adapt accordingly.

  We have some ideas about how this search might be improved by
  specifically looking for other rows that do not match the given
  condition, and where other database values are different, but
  did not have time to implement these by the project deadline.
  Instead, we leave these improvements as future work. In the
  meantime, developers can select an approach of their choosing, or use all 3
  for more exhaustive testing.

\item Django, and applications written atop it, use a significant
  subset of the operations and types provided by the Python
  language. The concolic execution framework only supports a
  subset of these (regular expressions and floating point numbers
  are noteworthy examples), and thus the concolic nature of
  values can easily get lost for today's complex web
  applications. We spent considerable time determining exactly
  which part of the Django codebase was using unsupported
  operations (e.g. URL resolution), and coming up with ways of
  circumventing those operations. Many of these were deeply
  integrated into the Django codebase, such as the parsing of
  POST form data and URLs. There are likely still parts of the
  Django codebase where concolic values would be lost, and
  testing the framework on a more diverse set of applications
  would be necessary to expose all of the corner-cases.

\item Web applications receive input from a number of sources; in-URL
  parameters, GET parameters, POST form parameters, file uploads,
  environment variables, and cookies just to name a few.
  Furthermore, there is no straightforward way of determining
  which of these a given application view uses, an in particular,
  which values it cares about. We have taken the approach of
  requiring the developer to explicitly construct the datasets
  (with concolic values where appropriate) to give to
  applications, and to pass these datasets in whenever a URL
  points to a relevant view. While this is inconvenient (the
  developer has to know and enumerate what data can be passed to
  every view of interest), it was the only feasible way we could
  imagine doing it in the available timeframe. Future work should
  include finding better ways of determining view inputs, or at
  the very least, simplifying the developer's job of telling the
  framework about the format of these inputs.

\item The framework from Lab 3 only tests invariants across single
  requests. Realistically however, a sequence of requests may be
  required for a particular bug to be exposed. We thus had to add
  support for running multiple requests one after another, while
  considering the entire sequence of requests a single execution.
  This is primarily hard because of the exponential growth of the
  search space, as the number of branches for each request are
  multiplied together. We chose to add this feature to the
  framework for the sake of completeness, but have not had time
  to test or optimize this feature for any of our applications.
\end{enumerate}


\subsection{Handling database queries}

We implemented 3 different approaches for accessing the database in a
concolic-compatible way. Each approach offers both positives and
negatives, and it seems unlikely that any solution will be superior
than all others in the general setting. An overview of our approaches:

\begin{enumerate}
\item \underline{All:}
  As in lab3, whenever we want to look up a value in the database, we
  first fetch every object, then compare each turn to the desired
  lookup key. We extend lab3's implementation to support multiple key
  lookups, and lookups across related tables.

  This approach has the benefit that it is relatively simple, and
  generally provides good coverage. The drawbacks are that it has the
  potential to re-test the same branches of the programs, which is
  wasteful. Further, there are situations (described below in
  \underline{SQL}) where branches will not be covered.

\item \underline{Mutation:}
  We implemented a mutation approach based on
  ConSMutate~(Sarkar, et.\ al. 2012). The goal is to catch typical
  programmer errors, e.g.\ writing a database query with
  $\textit{zoobars} > 10$ instead of \textit{zoobars }$ \geq 10$.

  Whenever the database is queried, we extract the WHERE clause and
  create mutated variants by changing the relational operator. We run
  each mutation and compare its result with that of the original. If
  the results are the same, we do nothing. If they are different, we
  add the symmetric difference of the two conditions as a new path
  constraint: hopefully this new constraint will lead to a concrete
  value that distinguishes the two conditions, and explores an
  untouched branch of the code.

  This approach has lower coverage but can potentially catch common errors.

\item \underline{SQL:}
  Consider the following pseudocode:
  \begin{verbatim}
  user = User.objects.get(username='bob')
  if user.person.zoobars > 5:
    # do something
  else:
    # do something else
  \end{verbatim}

  There are two issues here. First, once loaded from the database, the
  properties of the \textit{user} object, e.g.\ \textit{user.person.zoobars}
  will not have concolic values, so the if statement will simply compare
  two plain integers. In order to test both branches, we have to hope that
  in one iteration \textit{user} has more than 5 zoobars, and in another
  less than or equal to 5. There are no guarantees this will happen.
  Second, it may be the case that the database does not have a user
  with more than 5 zoobars and another with less than or equal to 5,
  in which case we have no hope of covering both branches, even by
  enumerating all rows, as in \underline{All}.

  We attempt to solve these issues in two ways:
  \begin{enumerate}
  \item Create new concolic values for each database object's
    properties. When \textit{user} is created above, we create a new
    concolic int for \textit{user.person.zoobars}. The if statement
    will then generate a new constraint, as desired, and Z3 can then
    produce a concrete value for us to test. Each concolic value in
    the system requires a unique ID, typically selected by the user.
    In this case, we create them automatically using a hash of the
    stack trace of the query as its unique identifier.

  \item Automatically create new database entries for concrete values
    that are not present. When Z3 generates a new value for
    \textit{user.person.zoobars}, say 0, it is possible that there is
    no user in the database with that value. So we create a new entry
    with this and any other concrete values.
  \end{enumerate}

  Testing showed that this approach provides slightly better coverage
  than enumerating all possible rows, while also requiring slightly
  fewer iterations, making it a promising solution. The downside is
  that it makes it more difficult to write program invariants, which
  typically rely on the database remaining unchanged.
\end{enumerate}

\subsection{Concolic optimizations}

Because our goal is to deploy concolic execution on large, real-world
Django web applications, we would like the constraint-checking framework
to be as efficient as possible. To that end, we implemented the
following optimizations to the Lab 3 framework in fuzzy.py and measured
their performance.

\begin{itemize}
\item When a request has been processed through the web app and a path
  condition is obtained, one identifies partial paths that branch off the
  current path and puts them on the queue. In Lab 3, we were branching
  both ``left'' and ``right'' on each element in the path. However, since
  one of these is going to be a partial path of the current one, it adds
  no new information. Not adding that path on the queue reduces the number
  of iterations by half. In Zoobar, we went from 181 to 89 iterations.

\item We noticed that the concolic engine often ran inputs it had
  already tried, even though we already maintain a cache of path
  conditions previously solved. It turns out different path conditions
  may yield the same solution, in which case they should be considered
  equivalent. The solver computation cannot be avoided, but one can
  eliminate processing an identical request in the web app. This further
  reduced Zoobar's iterations from 89 to 81.

\item The KLEE paper mentions removing redundant constraints from a
  constraint program to reduce the complexity for the solver. If a new
  constraint implies an existing one, or vice versa, then the constraint
  that is implied can be eliminated.
  Determining implication in the general case is expensive, requiring a
  solver. To avoid the cost of running Z3, we checked pairs of constraints
  against several patterns of string constraints where one is known to
  imply the other. We were able to detect all implications in Zoobar that
  Z3 would have, and further reduced the number of iterations to 75.

\item A novel feature of KLEE is its use of the counter-example cache
  that records previous solutions or no-goods, to be used as a heuristic
  on future iterations. The cache does not eliminate any iterations, but
  the runtime benefit of finding a solution in the cache may outweigh the
  cost of looking up the cache and having to run the solver anyway.
  The cache is implemented as a UBtree by [Hoffmann and Koehler],
  which allows efficient subset and superset queries. On Zoobar, the
  counter-example cache was successfully employed 28 times.
\end{itemize}

\end{document}
