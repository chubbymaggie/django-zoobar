\documentclass[a4page]{scrartcl}

\usepackage{fullpage}
\usepackage{url}

\author{Luke Anderson, Jon Gjengset, Jeevana Inala, and Andrew Wang \\
	\texttt{\{lukea,jfrg,jinala,wangaj\}@mit.edu}
}
\title{6.858 project proposal}
\subtitle{Concolic Execution for Django Applications}
\date{\today}

\begin{document}
\maketitle

\section{Project motivation}
The third lab exercise in 6.858 uses the z3 solver to implement concolic
execution for the Zoobar python application. Unfortunately, the framework used
for the lab is written specifically for Zoobar, and is unlikely to be directly
appliacble to other web applications. Having access to the power of concolic
execution could significantly improve a developer's ability to check crucial
security invariants in their web applications, and so we wish to extend the lab
3 concolic execution implementation to one that operate on applications that
are built on top of the popular Django web application framework.

\section{Deliverables}
\begin{enumerate}
	\item A rewrite of the Zoobar web application in Django.
	\item A modified version of Django that supports concolic execution.
	\item A polished and fleshed-out version of the lab3 concolic execution
		framework that supports Django web applications out of the box.
	\item (\textit{Wishlist}) Django ORM support plugin/patch for the
		Microsoft z3 solver.
\end{enumerate}

\section{Project stages}
First, we have to rebuild the Zoobar application functionality using the Django
framework. As Zoobar is a fairly standard web application, this is a relatively
small part of the project.

Next, we will instrument the Django source code so that it can support symbolic
execution of interesting input parameters. For most applications, this is
likely to be the request URL, the query parameters, the request type
(GET/POST/PUT/DELETE), and the cookie values.

After instrumenting Django, we will run the Zoobar application with concolic
execution, and make the modifications to the lab3 concolic execution framework
that are necessary to get it to run correctly on this new application.

If time allows, we also wish to implement support for the Django
Object-Relational Mapper (ORM) directly inside the z3 solver. This would allow
z3 to query the database directly for objects satisfying constraints (using
SQL) rather than relying on the developer enumerating all database records and
looking them up using native python operations as in Exercise 5. This should
substantially improve concolic execution time, and will make it much easier to
use our framework in complex applications. However, given the complexity and
magnitude of this task, we will only do it if the other components turn out to
be easier than expected and we are left with spare time.

\section{Challenges}
\begin{enumerate}
	\item Cookies in Django contain session IDs that are then used to look
		up the variables stored in that user's session. This is
		substantially more complex than the Zoobar cookies which simply
		included the username of the user that was logged in.
		Supporting these kinds of session cookies is vital to being
		able to test real web applications, and will likely be
		non-trivial.
	\item Modifying the lab3 concolic execution framework to support Django
		applications will probably include extending the set of
		supported operations supported quite substantially. This is
		because Django and ``real'' web applications probably exercise
		a larger subset of python operations.
	\item The complexity of Django, and many applications running atop it,
		means the state space can grow very large very quickly. It is
		likely that we will need to find and implement optimizations in
		the fuzzer that limit the concolic execution branching factor.
	\item Extending z3 to have support for database queries directly
		through the Django ORM requires digging through z3 to figure
		out how to translate symbolic conditions into SQL clauses.
		We will also need to find a way of inferring which symbolic
		variables in conditions correspond to database objects, and
		which are primitives (strings, ints, etc.). Solving this
		problem is likely going to require deep-seated and non-trivial
		changes to z3.
\end{enumerate}

\end{document}
